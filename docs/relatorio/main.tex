%%%%%%%%%%%%%%%%%%%%%%%%%%%%%%%%%%%%%%%%%%%%%%%%%%%%%%%%%%%%%%%%%%%%%%%%%%


% abnTeX2: Modelo de Trabalho Acadêmico em conformidade com 
% as normas da ABNT


%%%%%%%%%%%%%%%%%%%%%%%%%%%%%%%%%%%%%%%%%%%%%%%%%%%%%%%%%%%%%%%%%%%%%%%%%%


\documentclass[english, 
               brazil, 
               bsc] %Opções bsc (TCC) e msc (Mestrado)
               {dcomp-abntex2}




%%%%%%%%%%%%%%%%%%%%%%%%%%%%%%%%%%%%%%%%%%%%%%%%%%%%%%%%%%%%%%%%%%%%%%%%%%
% Área para adição de pacotes extras
%%%%%%%%%%%%%%%%%%%%%%%%%%%%%%%%%%%%%%%%%%%%%%%%%%%%%%%%%%%%%%%%%%%%%%%%%%


% \usepackage{lipsum} % Retirar para a versão final do documento
\usepackage{float}
\usepackage{pgfgantt}
\usepackage{lscape}


\restylefloat{table}


%Utilize aqui seu pacote preferido para algoritmos
\usepackage[linesnumbered]{algorithm2e}


%%%%%%%%%%%%%%%%%%%%%%%%%%%%%%%%%%%%%%%%%%%%%%%%%%%%%%%%%%%%%%%%%%%%%%%%%%


%Compila o índice
\makeindex


\begin{document}


% Seleciona o idioma do documento (conforme pacotes do babel)
\selectlanguage{brazil}


% Retira espaço extra obsoleto entre as frases.
\frenchspacing 


%%%%%%%%%%%%%%%%%%%%%%%%%%%%%%%%%%%%%%%%%%%%%%%%%%%%%%%%%%%%%%%%%%%%%%%%%%
% ELEMENTOS PRÉ-TEXTUAIS
%%%%%%%%%%%%%%%%%%%%%%%%%%%%%%%%%%%%%%%%%%%%%%%%%%%%%%%%%%%%%%%%%%%%%%%%%%


\pretextual




\titulo{PreOCR - Trabalho de Processamento de Imagens T01 2023.2} 
\autor{Grupo 13 - Everton Santos de Andrade Júnior}
\orientador{}
\coorientador{}


% \inserirInformacoesPDF





% \inserirInformacoesPDF
%
%
\imprimircapa
% \imprimirfolhaderosto*
%  
%     
\mostrarSUMARIO


%%%%%%%%%%%%%%%%%%%%%%%%%%%%%%%%%%%%%%%%%%%%%%%%%%%%%%%%%%%%%%%%%%%%%%%%%%
% ELEMENTOS TEXTUAIS
%%%%%%%%%%%%%%%%%%%%%%%%%%%%%%%%%%%%%%%%%%%%%%%%%%%%%%%%%%%%%%%%%%%%%%%%%%


\textual


%%%%%%%%%%%%%%%%%%%%%%%%%%%%%%%%%%%%%%%%%%%%%%%%%%%%%%%%%%%%%%%%%%%%%%%%%%
% Introdução
%%%%%%%%%%%%%%%%%%%%%%%%%%%%%%%%%%%%%%%%%%%%%%%%%%%%%%%%%%%%%%%%%%%%%%%%%%
\chapter{Introdução} \label{introduction}

Neste trabalho, propomos desenvolver um programa capaz de processar imagens binárias no formato PBM ASCII (PGM tipo P1), contendo texto em colunas, para determinar o número de linhas e palavras no texto. Além disso, nosso grupo implementou funcionalidades avançadas, como a detecção de blocos e colunas de texto, utilizando o conceito de distância alinhada.

Essa abordagem permite identificar estruturas de texto em diferentes alinhamentos, como justificado, esquerda, centro e direita, além de lidar com diferentes tamanhos e estilos de fonte, incluindo Comic Sans, Impact, Cascadia Code, Arial e Times New Roman. Demonstraremos como nosso programa é capaz de reconhecer blocos de texto em diversas configurações, desde fontes pequenas com tamanho 10 até fontes maiores com tamanho 40.

Adicionalmente, para ilustrar nossos resultados de forma mais interativa, geramos uma série de imagens intermediárias que mostram o processo de detecção de blocos e colunas, destacando as regiões de texto identificadas. Essas imagens proporcionam uma visualização detalhada dos algoritmos implementados e demonstram a robustez de nosso programa em lidar com uma variedade de casos de uso.

% Breve descrição do problema abordado no trabalho.
% Objetivo do trabalho.
% Descrição do Problema:

% Explicação detalhada do problema proposto.
% Especificações da entrada e saída do programa.
% Exemplos de imagens de entrada e saída.

\chapter{Métodos e Implementações}

% Descrição das técnicas utilizadas para resolver o problema.
% Explicação de como as técnicas aprendidas na disciplina foram aplicadas.
% Parâmetros utilizados durante o processamento das imagens.

% Implementação:
% Inclusão de código-fonte relevante ou detalhes adicionais sobre a implementação, se necessário.
Seguindo as formulas de dilatação em \cite[capitulo 9]{gonzalez2008digital}

\chapter{Resultados}

implementção \url{https://www.youtube.com/watch?v=uA45GeodGss}
Descrição da implementação do programa.
Destaque para soluções desenvolvidas para problemas específicos encontrados durante o desenvolvimento.
Resultados:

Apresentação dos resultados obtidos.
Inclusão de exemplos de imagens de entrada e saída.
Discussão sobre a eficácia do programa e eventuais limitações.


\chapter{Conclusão}

Sumarização dos principais resultados e contribuições do trabalho.
Reflexão sobre o aprendizado durante o desenvolvimento do programa.
Sugestões para trabalhos futuros ou melhorias no programa.




\phantompart
\bibliography{Bibliografia}


%%%%%%%%%%%%%%%%%%%%%%%%%%%%%%%%%%%%%%%%%%%%%%%%%%%%%%
% ELEMENTOS PÓS-TEXTUAIS
%%%%%%%%%%%%%%%%%%%%%%%%%%%%%%%%%%%%%%%%%%%%%%%%%%%%%%


\postextual


\renewcommand{\chapnumfont}{\chaptitlefont}
\renewcommand{\afterchapternum}{}
% \include{Pos_Textual/Apendices}
% \include{Pos_Textual/Anexos}


\end{document}
